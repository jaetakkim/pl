\documentclass[11pt]{article}
\usepackage[margin=1in]{geometry}
\usepackage{amsmath,amsthm,amssymb}
\usepackage{stmaryrd}
\usepackage{clrscode3e}
\usepackage{enumitem}
\usepackage{xspace}
\usepackage{tikz}
\usepackage{graphicx}
\graphicspath{ {./imgs/} }
\usepackage{underscore}
% \usetikzlibrary{matrix,arrows,decorations.pathmorphing,cd}
\setlist{
  listparindent=\parindent,
}
\usepackage{pgfplots}
\pgfplotsset{width=10cm,compat=1.9}
\usepackage{float}
\usepackage{polynom}
\usepackage{hyperref}
\expandafter\def\expandafter\UrlBreaks\expandafter{\UrlBreaks%  save the current one
  \do\a\do\b\do\c\do\d\do\e\do\f\do\g\do\h\do\i\do\j%
  \do\k\do\l\do\m\do\n\do\o\do\p\do\q\do\r\do\s\do\t%
  \do\u\do\v\do\w\do\x\do\y\do\z\do\A\do\B\do\C\do\D%
  \do\E\do\F\do\G\do\H\do\I\do\J\do\K\do\L\do\M\do\N%
  \do\O\do\P\do\Q\do\R\do\S\do\T\do\U\do\V\do\W\do\X%
  \do\Y\do\Z}
\newcommand{\qq}{\noindent\textbf{Q:} }
\newcommand{\N}{\mathbb{N}}
\newcommand{\Z}{\mathbb{Z}}
% various useful symbols defined here:
\newcommand{\np}{\textsc{NP}\xspace}
\newcommand{\conp}{\textsc{coNP}\xspace}
\newcommand{\p}{\textsc{P}\xspace}
\newcommand{\twosat}{\textsc{2-Sat}\xspace}
\newcommand{\threesat}{\textsc{3-Sat}\xspace}
\newcommand{\vc}{\textsc{VertexCover}\xspace}
\newcommand{\is}{\textsc{IndependentSet}\xspace}
\newcommand{\nphard}{\textsc{NP-hard}\xspace}
\newcommand{\npcomplete}{\textsc{NP-complete}\xspace}
\newsavebox{\wideeqbox}
\newenvironment{proof-idea}{\noindent{\bf Proof Idea}\hspace*{1em}}{\qed\bigskip}
\newenvironment{wideeq}
  {\begin{displaymath}\begin{lrbox}{\wideeqbox}$\displaystyle}
  {$\end{lrbox}\makebox[0pt]{\usebox{\wideeqbox}}\end{displaymath}}


\makeatletter
\renewcommand*\env@matrix[1][\arraystretch]{%
  \edef\arraystretch{#1}%
  \hskip -\arraycolsep
  \let\@ifnextchar\new@ifnextchar
  \array{*\c@MaxMatrixCols c}}
\makeatother

\newtheorem*{theorem}{Theorem}

\newtheorem*{prop}{Proposition}
\newtheorem*{conjecture}{Conjecture}

\theoremstyle{remark}
\newtheorem*{note}{Note}

\theoremstyle{remark}
\newtheorem*{remark}{Remark}

\newenvironment{must-show}{\noindent{\bf Must Show:}\hspace*{0.5em}}{\qed\bigskip}


\title{Zero Path Reachability}
\author{Jae Tak Kim}
\date{}

\begin{document}
\maketitle

\begin{enumerate}
    \item It's clear that every Z-path must be of the form $k$ number of $-1$ followed by $k$ number of $+1$, where $k\in\mathbb{N}$. Thus, every Z-path of length $2k$ contains within it Z-paths of length $2,\ldots,2(k-1)$. My initial idea was to start with Z-paths of length 2 and then build them up as large as you can in the graph, sort of like Prim's algorithm for MST's. However, I couldn't get an algorithm that runs in $O(EV)$-time.

    Another idea in a similar vein is to consecutively shrink the Z-paths from smallest to the largest. That is, to get every pair of consecutive $-1$ and $+1$ weight and combine them into one edge. If you have nodes $a,b,c,d,e$ and edges $(a,b), (b,c), (c,d), (d,e)$ with $w(a,b)=w(b,c)=-1$ and $w(c,d)=w(d,e)=+1$, you would first combine edges $(b,c)$ and $(c,d)$ into one edge $(b,d)$. However, you have to be sure not to remove the combined edges because there could be another Z-path if edge $(f,c)$ was in the graph with $w(f,c)=-1$, and if you removed $(c,d)$, the algorithm would incorrectly miss the Z-path from $f$ to $d$. The natural weight for the newly created edge is 0 since the weight of the path $p=b,c,d$ has weight 0. We could merely record this new edge in a separate list or adjacency matrix but since we know that this Z-path is a part of a longer Z-path, we want to use this information somehow to create a new edge $(a,e)$ of weight 0. Since we added the weights of the two edges to create $(b,d)$, we can do the same with the other edges. We can add the weights of edges $(a,b)$ and $(b,d)$ to create an edge $(a,d)$ of weight $-1+0=-1$, which would then be combined with the last remaining edge $(d,e)$.

    \begin{codebox}
    \Procname{\proc{CFGtoCNF}$(G=(N, \Sigma, R, S))$}
    \li \Comment Step 1: Create dummy non-terminals
    \li \For $(X\rightarrow Y) \in R$ where $len(Y) \geq 2$:
    \Do \li \If $\exists V \in Y$ where $V\in N$:
                \Then \li \Comment RHS should have at least one variable
                \li \For $\sigma \in Y$ where $\sigma \in \Sigma$:
                    \Do \li \Comment loop through terminals
                    \li \Comment create dummy variables
                    \li add new variable $V'$ to $N$
                    \li add rule $(V' \rightarrow \sigma)$ to $R$
                          \End
          \End
    \End
    \li \Comment Step 2: Convert unit productions
    \li \While $\exists (X \rightarrow Y) \in R$ where $len(Y) == 1$ and $Y[0]\in N$:
    \Do \li \For such $(X\rightarrow Y)$:
            \Do \li \Comment The single RHS must be a variable
            \li $newY \gets [ ]$
            \li \For $(X',Y') \in R$ where $X == Y[0]$:
                \Do \li add $Y'$ to $newY$
                \End
            \End
    \End
    \li \Comment Step 3: Make all rules binary
    \li \While $\exists (X\rightarrow Y)$ where $len(Y) > 2$:
    \Do \li \For such $(X\rightarrow Y)$:
            \Do \li \Comment Arbitrarily replace the first two variables
            \li Replace $Y[0:2]$ with new variable $V'$
            \li Add rule $(V'\rightarrow Y[0:2])$ to $R$
            \End
    \End
    \end{codebox}
\end{enumerate}

\end{document}
